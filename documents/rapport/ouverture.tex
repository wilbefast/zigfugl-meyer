    \section{Ouverture}
Nous avons vu %dans la partie Résultats (\ref{resultats})
 qu'il était effectivement possible de proposer un prototype offrant un aperçu des possibilités de l'utilisation d'un capteur comme le Kinect dans la réalisation de tests sensorimoteurs. Cependant, à la vue du certain nombre de contraintes que nous avons pu soulever en \ref{limites}, nous sommes arrivés à la conclusion qu'en l'état, le test de Fugl Meyer n'était pas le plus adapté. 

\paragraph{}
Cela amène à une réflexion, qui serait d'orienter un éventuel développement futur plutôt vers des exercices de rééducation ou d'autres types de tests que vers les exercices du test de Fugl Meyer. Comme nous avons pu le dire, il existe bien d'autres tests 
dans le domaine de la réhabilitation\,\footnote{voir l'Analyse de la littérature \textit{Evaluation of the disabilities of hemiplegic patients [M.-C. Gellez-Leman et al.]}}. Nous citerons notamment un test que nous a vanté J. Métrot lors de notre passage à 
Lapeyronie : ARAT.
\paragraph{Action Research Arm Test” (ARAT) \\}
ARAT est un test fonctionnel et non de déficience, se concentrant sur ce sur quoi le patient est capable de réaliser, et
qui a un véritable intérêt dans ses gestes quotidien. On chronomètre le temps mis pour ramasser des objets de diverses tailles,
pour mettre la main à la bouche (intérêt pour se nourrir) ou encore le temps mis pour transvaser de l'eau d'un verre à un autre.
\\ On notera que ce genre de test est beaucoup plus facilement intégrable dans une application de contrôle. Il suffit en effet de 
lancer un chrono et de valider si oui oui non le patient arrive à atteindre la consigne, sans considération pour des informations
comme ses tremblements ou des gestes parasites, compensations etc. Par ailleurs, les tests fonctionnels se prêtent beaucoup
plus à une éventuelle "gamification". Une interface graphique proposant au joueur/patient de réaliser des actions pré-établies (avec éventuellement des objets préalablement désignés) correspondant aux mouvements moteurs du test, semble offrir bien
plus de possibilités.

\paragraph{Interface Orale}
Une des remarques mise en avant était le risque de passer plus de temps à naviguer entre les différentes scènes de jeu/test à cause de la courte durée des exercices, et de leur ordre indéfini. Une solution pourrait être de proposer un système de 
commandes vocales qui permettraient de s'affranchir d'une GUI intermédiaire, et de changer aisément de scène.

\paragraph{Projets de stage} %TODO william je te laisse voir ça? -- c'est bon pour moi
Ce projet de TER peut s'inscrire comme la première démarche d'un projet à plus long terme pour les deux étudiants. Toujours
en coopération avec la société NaturalPad, qui se situe dans le secteur informatique de Serious Games à but 
thérapeutique, les deux étudiants vont en effet prolonger la démarche en réalisant chacun un stage chez NaturalPad. Brièvement,
le premier stage va consister à réfléchir aux moyens de réaliser une extraction de squelette de qualité, le second à lier les
paramètres physiologiques médicaux à des paramètres informatiques de gameplay dans le but de proposer un Serious Game
ludique et s'inscrivant complètement dans la démarche de réhabilitation sensorimoteur des patients / joueurs. 
%TODO   - possibilités de concurrence avec Zigfu vu qu'il a juste repris    
