\section{Solutions}
    
\subsection{Outils et technologies} 
\subsubsection{Pourquoi la Kinect?}
La Kinect est, d'après ce que nous venons de voir, un outil peu cher, 
très disponible car grand-public, et utilisable sur toutes les plateformes
majeures, avec des bindings pour la majorité des langages de programmation 
bien-connus.

\subsubsection{Pourquoi Unity?}
Nous avons eu quelques réticences par rapport à Unity, étant donné que c'est 
un outil fermé qui n'est utilisable que sous Windows et Mac. Pourtant nous ne 
pouvons nier les résultats obtenus avec en de très courtes durées. C'est
un outil parfait pour le prototypage rapide et, vu qu'il s'agit à la base d'un
moteur de jeu, il permet facilement d'ajouter des aspects jeu à notre projet.

\subsubsection{Pourquoi Zigfu?}
Étant donnée les choix de la Kinect et de Unity il n'y avait pas beaucoup d'options
pour la bibliothèque de middleware. Des binding OpenNI pour Unity existaient
à une époque, mais ne sont plus maintenus depuis 2 ans~: en effet c'est de ce
projet qu'est parti Zigfu pour OpenNI\cite{zigfu_branch}.

    
\subsection{Cahier des charges}
Ce projet de TER se trouvait à la liaison entre projet de recherche et projet d'application. Si une grande part a été accordée à 
l'étude de terrain, à l'acquisition des connaissances nécessaires et à une étude de faisabilité au moyen des technologies actuelles, un des objectifs principaux était aussi de proposer un prototype pour illustrer nos conclusions.
    
\subsubsection{Objectifs}
La première idée à bien retenir, est que nous voulions pouvoir apporter, ou montrer qu'il est possible d'apporter, une vraie plus 
value à la réalisation d'un test des capacités sensorimotrices avec une application utilisant un capteur comme le Kinect.
\paragraph{}
Naïvement, ces apports peuvent être multiples : gain de temps, gain de personnel, accroit de l'intérêt par le patient, appréciation
plus fine des résultats, panel de mesures plus important, visualisation de progression, détection de mouvements qui
échapperaient aux praticiens, etc.

\paragraph{}
L'objectif fut donc de créer une application fournissant un environnement graphique complémentaire pour le patient et le médecin.
Se voulant ensuite dérivable en un concept de jeu, il est important de noter les deux cibles distinctes de l'application. L'idée 
n'était pas de seulement fournie une interface au médecin lui proposant des informations supplémentaires, mais aussi de
permettre au patient de sortir du cadre strictement médical et mécanique que constitue la réalisation de son test.

\paragraph{}
Pour le médecin, on va vouloir lui communiquer des informations comme la valeur de position ou d'angle des membres du
patient concernés par l'exercice en cours. Cela afin de lui permettre d'évaluer plus aisément la réalisation du test. 

\paragraph{}
Pour le patient, on va souhaiter lui fournir une interface plus conviviale que les murs de l'hôpital ou du centre de soin et lui
proposer un système de feedback, lui permettant de juger lui-même de ses mouvements. Si cela peut sembler anodin, on
peut rappeler un sentiment désagréable qui peut exister dans le milieu médical, lorsqu'un patient ne comprend pas exactement
toutes les informations le concernant ou leur conséquences, ni parfois même ne les connait pas. Ainsi, l'idée est ici de lui
assurer un retour d'information immédiat, et il saura par exemple si il a échoué ou non à l'exercice.\\
On va aussi vouloir essayer de "gamifier" les exercices, en tentant d'apporter un environnement agréable, des objectifs
apparemment différents de ceux annoncés, un système de récompenses (comme un score) ou encore apporter du sens
à ses mouvements. Cela permet de sortir de la monotonie des test médicaux.

\paragraph{}
De manière plus commune, on va retrouver des éléments qui pourront être appréciés autant par le personnel médical que par 
le patient. \\
La première est la mise en place de graphes, qui vont afficher en temps réel l'évolution de différents paramètres en relation
avec l'exercice. Par exemple, on pourra afficher la courbe de rotation du coude, le niveau d'élévation de la main ou encore la
distance entre l'épaule et le poignet. Ce système de feedback peut apporter des informations décisives au médecin ou
illustrer au patient là où se situent ses difficultés.\\
On souhaitera aussi par exemple rappeler les valeurs attendues pour la réalisation du test, ou encore signaler la présence
d'un mouvement éliminatoire pour la réussite de l'exercice.

\paragraph{}
Le suivi des progrès semble aussi être un point crucial dans le domaine de la réhabilitation. Ainsi, la mise en place d'un score
plus fin que celui proposé par FM, ainsi qu'un système de suivi des capacités moteurs du patient nous semble tout à fait
adapté.

  % --- cela étant, les notions de feedback, d'intéractions 
    % 'meaningful play = integrated + discernable' (Rules of Play) les actions ont un impact et on sait lequel. 
    
    
\subsubsection{Choix d'implémentation}
Si nous espérions tout d'abord pouvoir réaliser une application permettant de contrôler la totalité d'un test de Fugl Meyer, il nous
est très vite apparu au fil de nos discutions et acquisition de connaissances, que cela ne serait pas possible. En effet, le test de FM
se révèle riche de plusieurs dizaines de mini-tests, et la plupart n'est en l'état actuel des technologies, pas mesurable efficacement (cf Limites \ref{limites}). Nous avons donc en premier lieu sélectionné les exercices (entendre, sous parties du test)
qui nous paraissaient les plus judicieux. Ceux-ci correspondent aux mouvements de :
\begin{itemize}
\item antepulsion de l'épaule : bras en avant 0\degre à 90\degre
\item flexion de l'épaule : bras en avant de 90\degre à 180\degre
\item abduction de l'épaule : bras sur le coté de 0\degre à 90\degre
\item dorsiflexion des poignets (2 positions : coude plié et bras tendu) : 0\degre à 15\degre verticalement
\end{itemize}

\paragraph{}
Ces exercices ont été choisis car seule la vision est nécessaire pour évaluer leur réussite ou non (pas besoin de contact direct).
De plus, la forte valeur des angles mesurés permet une marge de manœuvre appréciable, notamment au niveau de l'imprécision
éventuelle des mesures par le capteur. Ce sont des mouvements amples (sauf peut-être la dorsiflexion du poignet) et un bon
angle de vue permet au capteur de ne pas "perdre" les membres et ainsi fausser les résultats. Les angles mesurés ne
correspondent pas à la valeur réelle de l'angle de l'/des os du membre concerné, mais à l'angle créé par les segments représentés par les os des membres reliés. Par ex, on ne regardera pas si l'os de l'épaule (en réalité, il y en a 4) est à 90\degre, ce qui ne veut rien dire, mais la valeur de l'angle entre le bras et le tronc du patient.

\paragraph{}
Cette réflexion sur les angles est d'autant plus importante que le capteur Kinect ne peut actuellement pas mesurer directement
la valeur de rotation des différents membres, mais que l'on extrapole ces valeurs à partir des valeurs de positions des membres
reliés. Cela nous a donc fait exclure des exercices comme la prono-supination de l'avant bras : bras tendu, on présente successivement la paume puis le dos de la main en tournant son avant bras.

\paragraph{}
Nous avons aussi exclu de notre prototype les exercices mettant en jeu les membres inférieurs. La première raison est que nous n'avons pas assisté à une démonstration de telle exercices et aurions donc été moins à même de tenter d'en reproduire. La 
seconde raison est elle aussi d'ordre technique, puisque il est nécessaire de s'éloigner fortement du capteur Kinect pour que 
celui ci puisse détecter les jambes d'une personne. Cela implique d'avoir un grand espace de travail. Par ailleurs, nombre 
des exercices font assoir le patient, sur une chaise ou chevet, éléments qui perturbent la détection de la personne.

\paragraph{}
Notons enfin que les exercices impliquant les doigts et gestes étriqués ne conviennent pas non plus.