\documentclass[french,12pt]{report}
\usepackage[utf8]{inputenc}
\usepackage[francais]{babel}
\usepackage[T1]{fontenc}
\usepackage{lmodern}
\usepackage{ifpdf}
\usepackage{graphicx}
\usepackage{geometry}
\usepackage{color}
\usepackage{pdfpages}

\renewcommand{\familydefault}{\sfdefault}

\geometry{hmargin=50pt, vmargin=50pt}

\title{Rapport de projet : ZigFugl-Meyer}
\author{DYCE William \and MÉLIA Geoffrey}
\date{\today}
\ifpdf
	\pdfinfo 
	{
		/Author (wdyce, gmelia)
		/Title (Title (Rapport de projet))
		/Subject (Subject)
		/Keywords ()
		/CreationDate (\today)
	}
\fi

\begin{document}
	% Page de titre
	%\maketitle
	\thispagestyle{empty}
\begin{picture}(40,40)
\put(50,-600){\rule{.2mm}{21cm}}
\put(-50,-70){\rule{20cm}{.2mm}}

\put(70,-50){\textsc{\Huge{TER : ZigFugl-Meyer}}}
\put(360,-100){\textsc{\Large{Rapport de projet}}}

\put(100,-220){\textbf{\textsc{\large{Étudiants :}}}}
\put(100,-240){\textsc{\large{DYCE William, MÉLIA Geoffrey}}}

\put(100,-400){\textbf{\textsc{\large{Encadrant :}}}}
\put(180,-400){\textsc{\large{SEILLES Antoine}}}

\put(230,-530){\textsc{\large{Master informatique}}}
\put(205,-550){\textsc{\large{Année universitaire 2012-2013}}}
\end{picture}
	
	\thispagestyle{empty}
	\newpage
	
	% Sommaire
	\tableofcontents

	%Table des figures
	\listoffigures
	
	\newpage
	%TODO : grossir, centrer sur la page
	\section*{Remerciements}
	\addcontentsline{toc}{section}{Remerciements}
	\paragraph{}
	
\begin{itemize}
\item Antoine Seilles pour...
\item Ines Di Loreto
\item Julien Métrot et Karima Bakhti (et la jeune patiente) pour ...
\item Isabelle Laffont
\item Benoît Langes
\end{itemize}

\part{Rapport de projet}
\newpage
	\chapter{Introduction}
Les accidents vasculaires cérébraux (AVC) peuvent être la cause d'hémiplégie chez les personnes qui en sont victimes. La paralysie peut affecter une ou plusieurs parties du corps, jusqu'à être totale si la face, le tronc et les membres supérieurs et inférieurs sont paralysés. \\
La récupération des fonctions motrices, de la parole ou de la compréhension dépendent pour beaucoup de l'âge du patient et de son atteinte au niveau du cerveau.
		\section{Problématique}
Il existe de nombreux tests et échelles pour évaluer les capacités sensorimotrices de patients hémiplégiques [\ref {ref_analyse_litterature}]. Cependant, les médecins se retrouvent souvent confrontés au problème de la précision des mesures lors des test. Si la question ne se pose pas pour les tests dits "fonctionnels" (ex : le patient arrive t-il à se servir un verre d'eau?), des mesures d'angles et de positions se révèlent souvent nécessaires pour valider ou non la réussite d'un test par le patient.
\\Or, il existe peu d'outils pour réaliser ces mesures, et ceux-ci ne font pas l'unanimité. Pour la mesure d'angle entre les membres du patient (pli de l'épaule, du coude, etc.), le goniomètre se révèle être l'outil le plus utilisé, mais probablement par défaut (cf \ref{lapeyronie}). On lui reprochera en effet d'être: 
\begin{itemize}
	\item {intrusif :} Il doit être en contact direct avec le patient, pouvant fausser la mesure ou aider/gêner le patient.
	\item {imprécis :} L'épaisseur de peau et de graisse ne permet pas d'évaluer correctement l'angle formé par les os.
	\item {inutilisé :} Des médecins pourtant équipés vont préférer juger à l'œil, pour un meilleur ratio temps/précision.
\end{itemize}
		'une
		\subsubsection{Le test de Fugl-Meyer}
Cette imprécision des mesures devient dès lors gênante lorsque c'est un test non pas fonctionnel, mais de déficience qui est considéré comme le "Gold Standard" dans le domaine : le score de Fugl Meyer [\ref{fugl_meyer}]. En effet, celui-ci consiste à évaluer les mouvements du patient lors de la réalisation de gestes très précis, incluant des mouvements "éliminatoires", qu'il faut donc être capable de mesurer.	
	
	L'idée s'est alors posée d'utiliser un autre moyen de mesure, moins intrusif, pour la réalisation de tests de Fugl Meyer, largement utilisé dans le milieu de la réhabilitation, et de réfléchir aux possibilités d'enrichissement de celui-ci.
\newpage
		\section{Présentation du projet}
		
			\subsection{Étude de terrain} \label{etude_terrain}
		Le projet est avant tout une étude de faisabilité. Nous essayerons de voir
		à quel point nous pouvions procéder, avec un capteur de profondeur (en
		l'occurrence la Microsoft Kinect) et des bibliothèques et outils couramment 
		disponibles, à un test complet ou partiel de Fugl-Meyer.
		
		Nous étudierons donc d'une part les technologies disponibles pour la Kinect,
		et d'autre ce qu'est le test de Fugl-Meyer et quelles sont les 
		problématiques associées.
		
		\subsection{Prototype}
		Nous élaborerons ensuite une preuve de concept,données de tout c'est à dire un 
		prototype proposant au patient en cours de réhabilitation un test partiel 
		de Fugl-Meyer. Les points importants à considérer pour un tel prototype 
		sont~:
		\subsubsection{Automatisation et objectivité}
		Le système doit pouvoir procéder à une évaluation, de manière non-intrusive 
		et sans intervention externe, de ce qui se trouve devant lui. Il doit donc 
		proposer une série de notes objectives qui correspondent aux performances du 
		patient.
		\subsubsection{Aspect ludique~: "gamification"}
		La "gamification" est le processus qui consiste à ajouter des éléments 
		souvent associés au domaine du jeu (d'où le nom), comme des barres de
		progression ou des défis à relever, à une activité non-ludique. L'idée est 
		de favoriser l'implication et la motivation des participants. 
		Notons qu'il ne s'agit pas nécessairement de faire de l'activité en soi un 
		jeu.
    \subsubsection{Retour d'information continue et granulaire}
    La "granularité" d'un retour est le nombre d'options possibles. 
    Chaque exercice du Fugl-Meyer, par exemple, est noté sur trois points. Nous 
    aimerions proposer une alternative plus nuancée.
    L'argument pour le retour d'information rejoint celui de la "gamification"~:
    la notion de "sens" dans un jeu, au moins d'après Salen et Zimmerman, est 
    directement lié à la richesse de ce retour.
    
    % "Rules of Play" on "meaningful play"

	\chapter{Background}
		\section{Kinect}

			\subsection{Description}
La Kinect, connue à l'origine sous le nom "Project Natal" est une
périphérique informatique créé par Microsoft, comportant une caméra
couleur, un étalage de 4 microphones, des moteurs pour s'orienter avec 
accéléromètres 
et un système senseur de profondeur composé d'un 
projecteur infra-rouge et d'un capteur. 

L'appareil n'utilise pas, comme on pourrait croire, le "Temps de Vol" pour
calculer la profondeur d'une scène. C'est en fait la déformation de chaqu'un de 
sa matrice de rayons infra-rouge,
lorsqu'ils frappent une surface, qui est mesuré. Cette technique dite de 
"Lumière Structuré"
lui permet de générer une image 3D d'une scène à une résolution de 640x480 avec 
une profondeur de 11 bits, qui sera mise à jour 30 fois par seconde.

			\subsection{Historique}
Sorti en Novembre 2010, le système était prévu à l'origine pour la console
de jeu de Microsoft, la "Xbox360". Comme la plupart des consoles et
périphériques de jeu elle
est donc vendu à perte, l'idée étant de récupérer l'argent perdu en vendant
des jeux. Ceci en fait l'appareil de son genre le moins cher et le plus couramment
disponible.

Il est donc peu surprenant que, dès sa sortie, des pilotes Libres et gratuits
ont été créés afin de permettre le détournement du périphérique par des
chercheurs et des amateurs. Microsoft a répondu en début de 2011 avec un
premier SDK pour Windows initialement peu complet. Des mises à jours et une
deuxième version, sortie en Octobre 2012, comblent ces lacunes.

		
		\section{Aspect médical} \label{lapeyronie}		%---GEOFFREY---
		% je commente, car je ris à chaque fois que je lis cette ligne :
%		PARLER DES DUDES QU'ON A RENCONTRE A PERLARONIE
Ce projet, d'un choix assumé par les étudiants et la société NaturalPad qui en est l'inspiratrice, comporte une composante recherche forte. C'est dans cette optique et comme expliqué en \ref{etude_terrain} que nous avons commencé notre projet par un travail d'investigation et de recherche d'informations. Dans ce but, nous avons rencontré des praticiens de l'hôpital Lapeyronie à Montpellier : Julien Métrot, doctorant du Laboratoire Movement To Health et Karima Bakhti, kinésithérapeute au CHU de Montpellier. Lors de cette entrevue, ils ont partagé avec nous leurs connaissances sur le test de Fugl Meyer et bien plus encore, et nous ont offert la possibilité d'assister à la réalisation d'un tel test sur une patiente atteinte d'hémiplégie.
	\subsection{Fugl Meyer}
Le test de Fugl Meyer constitue un ensemble d'exercices à effectuer par le patient, et dont chaque mouvement est évalué selon un ou plusieurs critères, fournissant un score généralement compris entre 0 [échec], 1 [intermédiaire] et 2 [réussite]. Ces exercices peuvent concerner successivement les membres inférieurs et supérieurs, eux même divisés en sous catégories. Pour les membres supérieurs, on notera la présence d'un score \textit{proximal} et d'un score \textit{distal}. Ce test a pour vocation d'être réalisé régulièrement tout au long de la période de réhabilitation du patient, afin de mesurer ses progrès.
	\subsection{Réalisation du test}
				\subsubsection{Contexte}
Nous avons assisté à la passation d'un test de FM sur une jeune patiente atteinte d'une hémiplégie suite à un AVC ; ses membres supérieurs droits étaient affectés. Comme nous l'ont précisé J. Métrot et K. Bahkti, seules les parties du test concernant les membres supérieurs ont donc bien sur été réalisées. Pour les détails des mesures et exercices effectués lors de ce test, se référer à l'annexe \ref{evaluation_FM}.

				\subsubsection{À noter}
Les patients concernés par la rééducation des membres supérieurs, comme la patiente qui a accepté de passer 
le FM en notre présence, sont globalement plus nombreux. Cela concorde avec les contraintes du capteur Kinect, 
qui possède un petit angle de vue. Il est alors nécessaire d'avoir beaucoup d'espace (de recul) dès lors que 
l'on souhaite capter l'intégralité d'une personne de manière précise. \\
Lors des tests, le médecin ou personnel médical chargé s'occupant du patient, peut être amené à 'aider' celui-ci.
Cela peut se traduire, outre les éventuels encouragements oraux, en maintenant ou en bloquant un membre du patient
non directement concerné par l'exercice du test. Par exemple, lorsque l'on demande au patient de "tourner son avant 
bras" (prono-supination de l'avant bras) avec le bras tendu, le praticien peut aider le patient à maintenant son bras
(partie haute).\\
Notons aussi qu'en l'occurrence, les praticiens n'ont pas eu besoin d'utiliser de goniomètre ou autre instrument de 
mesure pour évaluer la qualité des gestes de la patiente.
		
	\subsection{L'avis des professionnels}
	Le test de Fugl Meyer est le plus répandu et accepté : il constitue le "gold standard" dans le milieu.
	Nos interlocuteurs regretteront toutefois une certaine subjectivité et interprétation des résultats.
	Nous en avons d'ailleurs eu une démonstration lors de ce test : il arrivait que K. Bahkti et J. Métrot 
	évalue différemment les mouvements de la patiente. \\
Une autre critique est "l'effet plateau" qu'induit ce test. En effet, l'échelle de mesure est très petite 
(2, maximum 3 valeurs de notation) et le patient atteindra régulièrement des 'seuils' sur son score global. 
C'est-à-dire que même si il réalise encore des progrès par rapport aux fois précédentes, cela ne se pas forcément 
visible et impactant sur le score global, à cause du manque de granularité de la mesure. Même si cela n'est pas une 
fin en soi, on peut noter l'aspect peu encourageant que cela peut apporter au patient. \\
Pour les mêmes raisons, le patient pourra obtenir un score maximal (dans notre étude de cas, 66/66) sans pour autant
avoir recouvré la totalité de ses capacités, et quand bien même des progrès sont encore possibles. \\
Cela amène au regret que le test standard soit un test jugeant la \textbf{déficience} et non la \textbf{fonctionnalité}. 
Il existe cependant de nombreux autres tests, et une des alternatives intéressante est le test "ARAT : 
Action Research Arm Test". %TODO réf lien
Rapidement, ce test consiste à vérifier si un patient est capable d'effectuer un geste du quotidien (déplacer
un objet, se servir un verre d'eau, etc), et de chronométrer le temps que cela lui a pris. C'est donc un test 
de fonctionnalité.
	
	\subsubsection{Impressions et propositions}
Après notre entretien et cette passation du test de FM, et à partir de nos tests préliminaires, nous en avons 
conclu que l'utilisation du kinect semblerait plus adpatée pour faire une évaluation des performances “proximales”.
En effet, la taille des membres et l'amplitude des gestes requis pour l'évaluation proximale offrent de bien 
meilleures possibilités d'analyse des mouvements. Certains gestes du poignet semblerait aussi bien adaptés.
\paragraph{}En revanche, les autres mouvements distaux semblent trop difficile à percevoir et interpréter par le capteur.
L’idée serait de proposer une évaluation de grain plus fine, allant au délà d’une évaluation trinaire entre 
“a réussi à lever le bras à l’horizontal”, “a levé son bras partiellement” et “n’a pas réussi à lever son bras”. 
Il serait utile de pouvoir donner l’angle exact, par exemple.   

\paragraph{}Les tests sur la main sont évalués de manière “tactile” (tests de résistance, de préhension) pour la plupart, 
avec peu de différence visuelle entre une évaluation 0 et 2. L’utilisation de la Kinect semble alors peu pertinente
pour cette partie de l’évaluation.
		
	\chapter{État de l'art}
	
	
  \section{Bibliothèques Kinect}
		
Il est important de noter que la reconnaissance vocale et l'extraction de 
squelette ne sont pas faits par l'appareil, ou "serveur", lui-même~: ce sont des
bibliothèques sur la console ou l'ordinateur, le "client", qui s'en chargent. 
Il faut donc bien
différencier les pilotes de bibliothèques de bas niveau, qui permettent 
d'accéder aux données brutes, des
bibliothèques de haut-niveau, qui interprètent ces données.

  \subsection{Pilotes et bindings bas-niveau}
  Le jour de la sortie de la Kinect, le 4 Novembre 2010, Adafruit Industries 
  offre un prix de \$2000 (augmenté ensuite jusqu'à \$3000) pour le premier qui
  sortira des pilotes permettant l'utilisant de la Kinect sur ordinateur.
  %https://www.adafruit.com/blog/2010/11/04/the-open-kinect-project-the-ok-prize-
  %get-1000-bounty-for-kinect-for-xbox-360-open-source-drivers/
  \subsubsection{Libfreenect}
  Le prix est remporté 6 jours plus tard le 10 Novembre par un programmeur Linux
  du connu sous le nom "Hector". C'est la première version de Libfreenect autour
  duquel le groupe OpenKinect se forme. Libfreenect a pour comme avantage son
  très grand nombre de bindings~: C/C++, Java, Python, Actionscript, Javascript,
  LISP, C\#, pour ne nommer qu'un sous-ensemble. 
  Il est toujours maintenu (dernier commit Décembre 2012).vu
  que leur API est le standard pour ce genre d'appareil 
  %https://www.adafruit.com/blog/2010/11/10/we-have-a-winner-open-kinect-
  %drivers-released-winner-will-use-3k-for-more-hacking-plus-an-additional-2k-goes-to-the-eff/
  \subsubsection{CL NUI}
  9 jours après la sortie de Libfreenect la compagnie Coding Laboratories 
  (CL) et le groupe logiciel-libre
  Natural User-Interaction (NUI) sortent CL-NUI. Nous sommes alors 2 semaines 
  après la sortie de l'appareil. CL-NUI propose des bindings 
  WPF\footnote{Windows Presentation Foundation}/C\# C et C++ et permet d'accéder 
  à tout sauf le microphone. Le développement continue jusqu'en fin Décembre 2010 
  mais,
  le défi déjà remporté, s'arrête par la suite. Notons pourtant qu'il faudra 
  attendre 2 ans pour avoir accès aux données des accéléromètres à travers la 
  SDK officiel\ldots
  \subsubsection{Kinect for Windows (Pilotes)}  
  Les pilotes officielles de Microsoft viennent avec leur SDK, "Kinect for 
  Windows". Inutile de dire
  qu'ils ne fonctionnent que sous Windows. Ce qui est un peu plus surprenant 
  c'est qu'il n'y a de soutient que pour Window 7 et 8, donc pas pour Windows XP 
  ou Vista.
  \subsubsection{SDK OpenNI (Pilotes)}
  OpenNI (Open Natural Interaction) est un consortium à but non lucratif composé
  des grands acteur économiques du domaine. S'y figure notamment PrimeSense, la 
  compagnie qui a développé la Kinect pour Microsoft au premier lieu. Leur SDK
  est plus ou moins analogique à celui de Microsoft où il est question de
  Kinect, mais fonctionne aussi sur les plus anciennes 
  versions de Windows mais
  aussi sous Linux (y compris Android) et Mac OSX.
  
  La OpenNI permet également de s'abstraire du matérielle utilisé. Il suffit que
  les pilotes utilisés soient conformes à ses standards~: pour passer, par 
  exemple, de la Kinect à une caméra PrimeSense nous devons simplement remplacer 
  les pilotes "SensorKinect" avec "PrimeSensor".
  
% ------------------------------------------------------------------------------  
  
  \subsection{Middleware}
  Les bibliothèques dites "middleware interaction naturelle" servent ici en 
  générale à faire une 
  extraction de squelette et une reconnaissance de gestes. Elles permettent donc 
  au développeur de s'abstraire des données brutes. Le groupe OpenNI défini 
  des normes auxquelles doivent obéir cette couche, de 
  manière à faciliter la communication à travers la pile de technologie.
  
  \subsubsection{Kinect for Windows (Middleware)}  
  Bien que les SDK de Microsoft et de la OpenNI se disent tout et deux des 
  "Simple Development Kit",
  celui de Microsoft, "Kinect for Windows", contient à la fois les pilotes, 
  l'API de bas niveau et une couche d'analyse des données qu'OpenNI appellerait 
  middleware. Il n'est donc pas fait pour interfacer avec les couches de la 
  OpenNI et donc ne suit par leur normes.
  
  Kinect for Windows a quelques avantages par rapport à NITE~: il propose un
  placement prédictive des articulations et genère deux points en plus, 
  le poignet et le bout des doigts. Cela étant sa prédiction consomme beaucoup 
  plus de ressources et peut amener à de faux positives.
  % http://labs.vectorform.com/2011/06/windows-kinect-sdk-vs-openni-2/
  
  \subsubsection{NITE de PrimeSense}
  NITE (Natural Interface Technology for End-User) est un framework Libre et 
  multiplateforme développé par PrimeSense. Vu que c'est PrimeSense qui a 
  lancé OpenNI, NITE est devenu la tête d'affiche de ses
  middleware standardisé. Elle se repose donc sur la SDK de la OpenNI. 
  \subsubsection{OpenCV}
  OpenCV (Open Computer Vision) est une vaste bibliothèque de toutes sortes de 
  fonctions aillant un rapport avec la détection de formes à partir d'images
  statiques ou dynamiques. Il existe depuis 1999, donc bien avant la Kinect.
  
  Étant donnée qu'il s'agit d'une bibliothèque généraliste il a peu de 
  fonctionnalités qui ciblent précisément les données 3D, contrairement aux 
  bibliothèques dites "d'interaction naturelle".
  
% ------------------------------------------------------------------------------  
  
  \subsection{Bibliothèques de haut-niveau}
  
  \subsubsection{Zigfu}
  Crée en May 2011, Zigfu est une petite entreprise formée de 4 personnes dont 
  Amir Hirsch, le fondateur, et 2 ex-employés de PrimeSense. Il fournit, 
  par dessus de l'extraction de squelette et de la reconnaissance
  de gestes, un API permettant de créer des boutons, listes, curseurs et autres
  éléments GUI, tous contrôlés par le mouvement.
  
  Zigfu intègre également la cinématique inverse et un système de 
  prédiction et
  d'interpolation à fin d'avoir une squelette répondant à des contraintes 
  physiques à 60Hz. La plupart des jeux moderne tournent à cette vitesse, et 
  rappelons nous que la caméra ne fournit que des données 30Hz
  
  %http://www.headlondon.com/our-thoughts/technology/posts/microsoft-kinect-and-the-web-browser-enter-zigfu
  %https://www.youtube.com/watch?v=uSo0jXYm5wQ
  %https://github.com/OpenNI/UnityWrapper
  %http://github.com/tinkerer/UnityWrapper

  \subsubsection{FAAST}
  La Flexible~Action~and~Articulated~Skeleton~Toolkit (FAAST) est une 
  bibliothèque qui se repose soit sur Kinect for Windows, soit OpenNI et NITE.
  Il n'est pour l'instant compatible qu'avec Windows.
  
  Développé par la Interactive~Media~Division (MxR) de 
  l'Institute~of~Creative~Technologies (ICT) de 
  l'University~of~Southern~California (USC). Il permet de monter à un niveau 
  encore plus abstraite~: l'utilisateur travail avec des événement générés à
  partir de postures et des gestes, au lieu d'analyser manuellement la 
  squelette.
  
% ------------------------------------------------------------------------------  
% ------------------------------------------------------------------------------  
  
  \section{"Interaction naturelle" pour l'évaluation}
		
  \subsection{Applications purement ludiques}
  
  \subsubsection{Wii Fit}
  \subsubsection{Nike+ Kinect Training}
  
% ------------------------------------------------------------------------------  
  
  \subsection{Applications thérapeutiques}

  \subsubsection{Moteur de Jeux Orientés Santé (MOJOS)}
  Voracy Fish
  % http://www.mojos.fr/home/
  
  \subsubsection{Hammer \& Planks}
  
  \subsubsection{Wii rehabiliation ??? }
	
% ------------------------------------------------------------------------------  
% ------------------------------------------------------------------------------  
% ------------------------------------------------------------------------------  
	
	\chapter{Propositions}
	
	
	
		--> WHERE PUT DIS?????? Faire un ***SERIOUS*** jeu
		
		
		
		\section{Solutions}
		
		\subsection{Outils et technologies} 		%---WILLIAM--
		

		
		- pourquoi la Kinect (?)
		-- c'est pas cher (trouver des exemples de trucs)
		-- c'est non intrusive (pas besoin de porter des trucs)
		-- marche sur les PC (pas besoin de HW dédié -- ça revient un peu dans le 'pas cher')
		
		- pourquoi Unity (?)
		-- pour la gamification
		-- outil de prototypage rapide
		
		- pourquoi Zigfu (?)
		-- intégration facile avec Unity
		-- pas beaucoup de choix (basé sur le projet de l'OpenNI pas mis à jour depuis 2 ans maintenant...)
		--- <- include: délire sur pourquoi ce projet a été abandonnée (EVENTUELLMEENT)
		
		
		\subsection{Cahier de charges / user stories} 		%---GEOFFREY--
		
		MAINTENANT QU'ON A FINI NOTRE ETUDE DE TERRAIN ON PROPOSE L'APPLICATION PARFAITE 
		
		ou pas - m'enfin bref.
		
		- faire les 4 exerices dont on a parlé
		-- pourquoi ces 4 là? Pourquoi pas les autres?
		--- problème pour les jambes (distances, bizarreté)
		--- problème pour mesurer la rotation (extrapolés)
		--- problème pour mesurer les doigts 
		
		- fournir un 'score' plus précis que celui que propose la FM
		-- l'angle exacte auquel on est arrivé
		-- savoir combien on a progrésé
		
		- tester si le patient fait un mouvement 'illegal'
		-- le prevenir
		-- recommencer le teste
		
		
		- gamification
		-- environnement agréeable
		-- ??? pourquoi ça s'applique mal
		--- mieux les truc répétifs (le FM on le fait genre une fois par semaine!)
		--- cela étant, les notions de feedback, d'intéractions 
			'meaningful play = integrated + discernable' (Rules of Play) les actions ont un 
			impact et on sait lequel. 
		
		\section{Mise en oeuvre} 	%	---WILLIAM + GEOFFREY--

		Grapher
		- FailGrapher (montre la croix en plus de la progression)
		
		Multicamera
		
		ExerciseMonitor
		- FlexionMontior et compagnie
		
		Machine à états (on aime bien les trucs visuels dans les rapports)
		
		\section{Limites}%		---GEOFFREY--
			\subsection{Test de Fugl Meyer}
Comme indiqué précédement, ce test constitue le standard en la matière et on ne peut donc pas passer à coté en premier lieu.
Cependant, les exerices et le test en lui-même ont pour vocation de rapidement évaluer les capacités ou déficience
du patient. La plupart des exercices du test ne dure ainsi que quelques secondes. Cela, couplé au fait que tous les 
médecins ne font pas faire chaque partie du test, ni surtout dans le même ordre, implique une interface permettant de 
choisir quelle partie du test exécuter. Le problème que l’on peut voir, c'est qu'au final on passerait 
bien plus de temps dans cette interface que dans les "jeux" à proprement parler.

\paragraph{}
D'après notre observation de la réalisation d'un test avec une jeune patiente, celle-ci focalise 
beaucoup son regard sur les membres concernés par l'exercice. Il nous a été confirmé que c'était un fait fréquent
chez les patients, notamment en début de réhabilitation. Pour cette raison, il n'est pas possible de leur
demander de regarder un écran tant qu'ils exécutent un mouvements leur demandant un effort particulier.
On retrouve aussi un test qui demande au patient de fermer les yeux.

\paragraph{}
Par ailleurs, les tests distaux (mains et poignets) nécessitent une précision que ne permet pas la kinect.
Pour les mains, les test sont évalués de manière “tactile” (tests de résistance, de préhension) pour la plupart, 
avec peu de différence visuelle entre une évaluation 0 et 2. L’utilisation de la Kinect semble alors 
peu pertinente pour cette partie de l’évaluation..

\paragraph{}
On rappellera le problème de "l'effet plateau", du à un manque de granularité de l'évaluation, et la possibilité
d'obtention d'un score maximal sans pour autant avoir atteint une guérison complète. Rappelons aussi la variabilité
des évaluations par les praticiens, et le problème que cela implique de trouver une manière "neutre" d'interprêter
les mouvements du patient.
Enfin, les gestes demandés par FM sont très précis et non modifiables. Cela peut limiter les possibilités de “gamification”.
		
			\subsection{Zigfu}	
		Zigfu c'est de la merde!!!!
		- ils profitent d'un projet open source
		- ils apportent rien
		- pas de docs
		- code mal écrit
		- la moitié des samples marchent pas pour WILLIAM :D
		
		\subsection{Capteur Kinect}
		La Kinect en générale ??????  et un apport du Kinect pas évident.
		- les rotations (pas récupérés par la SDK de Microsoft => déduits)
		  -- est il possible de les "voir" directement?  
Notons aussi que certains exo du test du Fugl Meyer ne peuvent pas être fait/interprêtés par la kinect, il en résulte
un test du Fugl Meyer tronqué et plus long à réaliser.
		- les mouvement distaux (pas de capture des doits individuelles)

\paragraph{Remarque :\\}
Chose à prendre en compte : ne pas imaginer un feedback "trop incitateur". Cad : si un patient n'arrive pas à 
réaliser un mouvement complet, le feedback ne doit pas l'inciter à forcer + qu’il peut raisonnablement le faire.
		
		\section{Ouverture}	%	---GEOFFREY--
		
		- alternatifs: ARAT = test fonctionnel et non de déficience (le médecin il le vendait TROP)
Cela amène à une reflexion, qui serait de plutôt orienter un éventuel développement futur (hors cadre TER) vers
des exercices de rééducation ou autres types de test plutôt que vers les 'exercices' du test de Fugl Meyer.
		- projet de stage avec NaturalPad: faire de extraction de squelette moins nul
		- possibilités de concurence avec Zigfu vu qu'il a juste repris 
		- la gamification est plus pertinente / adapté aux exercices de rehabiliation 
		-(Idée : interface orale?)
		
		\section{Références}
			\paragraph{Ressources web}
				\begin{itemize}
					\item zigfu.com
				\end{itemize}
			\paragraph{Articles}
				\begin {itemize}
					\item : Evaluation of the disabilities of hemiplegic patients [M.-C. Gellez-Leman et al.] \label{ref_analyse_litterature}
				\end{itemize}
				
\part{Annexes}
	\chapter{Fiche d'évaluation du score de FM pour les membres supérieurs} \label{evaluation_FM}
	%TODO \includegraphics[scale=1]{scan_fichie_evalation.jpngif}
\end{document}
