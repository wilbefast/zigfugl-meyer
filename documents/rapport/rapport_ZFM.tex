\documentclass{report}
\usepackage[utf8]{inputenc}
\usepackage[francais]{babel}
\usepackage[T1]{fontenc}
\usepackage{lmodern}
\usepackage{ifpdf}
\usepackage{graphicx}
\usepackage{geometry}
\usepackage{color}
\usepackage{pdfpages}

\renewcommand{\familydefault}{\sfdefault}

\geometry{hmargin=50pt, vmargin=50pt}

\title{Rapport de projet : ZigFugl-Meyer}
\author{DYCE William \and MÉLIA Geoffrey}
\date{\today}
\ifpdf
	\pdfinfo 
	{
		/Author (wdyce, gmelia)
		/Title (Title (Rapport de projet))
		/Subject (Subject)
		/Keywords ()
		/CreationDate (\today)
	}
\fi

\begin{document}
	% Page de titre
	%\maketitle
	\thispagestyle{empty}
\begin{picture}(40,40)
\put(50,-600){\rule{.2mm}{21cm}}
\put(-50,-70){\rule{20cm}{.2mm}}

\put(70,-50){\textsc{\Huge{TER : ZigFugl-Meyer}}}
\put(360,-100){\textsc{\Large{Rapport de projet}}}

\put(100,-220){\textbf{\textsc{\large{Étudiants :}}}}
\put(100,-240){\textsc{\large{DYCE William, MÉLIA Geoffrey}}}

\put(100,-400){\textbf{\textsc{\large{Encadrant :}}}}
\put(170,-400){\textsc{\large{SEILLES Antoine}}}

\put(230,-530){\textsc{\large{Master informatique}}}
\put(205,-550){\textsc{\large{Année universitaire 2012-2013}}}
\end{picture}
	
	\thispagestyle{empty}
	\newpage
	
	% Sommaire
	\tableofcontents

	%Table des figures
	\listoffigures
	
	\newpage
	%TODO : grossir, centrer sur la page
	\section*{Remerciements}
	\addcontentsline{toc}{section}{Remerciements}
	\paragraph{}
	
\begin{itemize}
\item Antoine Seilles pour...
\item Ines Di Loreto
\item Julien Métrot et Karima Bakhti (et la jeune patiente) pour ...
\item Isabelle Laffont
\item Benoît Langes
\end{itemize}
	\part{Rapport de projet}
	\newpage
	\chapter{Introduction}
		\section{Problématique} --GEOFFFREY--
		
		SUBSECTION THIS SHIT
		gonomètre (ici ou dans état de l'art????????????????)
		- intrusive
		- imprécis
		- la plupart de skinés préfèrent faire à l'oeil
		
		le test de Fugl-Meyer
		- "effet plateau" - manque de granularité, possibilité d'avoir tous les points sans être "guérri".
		- variabilité des mesures selon le medecin (même les deux qu'on a vu...)
		
	\newpage
		\section{Présentation du projet} 		---WILLIAM---

		Faire un test avec la Kinect - lol!
		
		étude de terrain + faisibilité -- INSISTER SUR CELA!!!!
		- kinect / zigfu
		- unity
		- fugl meyer
		
		développement d'un prototype:
		plus de précision
		plus de granularité
		meilleur suivie du patient (son progrès, etc)
		éventuelle gamification (expliquer après pourquoi on a laissé tomber)
		
		
	\chapter{Background}
		\section{Kinect} 		---WILLIAM---
		
		c'est un camera RGB + IR. C'est Miscrosoft qu'il l'ait fait - lol!
		
		\section{Fugl Meyer} 		---GEOFFREY---
		
		PARLER DES DUDES QU'ON A RENCONTRE A PERLARONIE
		
		
		c'est un test qui teste des choses.
		
	\chapter{État de l'art} 		---WILLIAM---
	

	
		\section{Biblio Kinect}
		
		Drivers:
		MS drivers
		libfreenect
		
		Bibliothèques:
		OpenNI
		SDK Microsoft
		NITE de Primesense (racheté par MS ? )
		OpenCV
		FAAST
		Zigfu
		... etc
		
		
		\section{Interface naturelles appliqués aux mesures de mouvement 
		(ce titre, c'est de la merde - je vais le changer, mais plus tard...}

		NI + REHAB
		Voracy Fish (MOJOS)
		les truc qu'avec fait le LIRMM (MOJOS)
 		Natural Pad bien sur! Hammer \& Planks.
		-- exemples de Wii rehabilitation ?
		
		NI + MESURE DE PERFORMANCE
		Nike Kinect Training
		Wii Sport / Fit
		etcetera etcetera
		
		
	\chapter{Propositions}
	
	
	
		--> WHERE PUT DIS?????? Faire un ***SERIOUS*** jeu
		
		
		
		\section{Solutions}
		
		\subsection{Outils et technologies} 		---WILLIAM--
		- pourquoi la Kinect (?)
		-- c'est pas cher (trouver des exemples de trucs)
		-- c'est non intrusive (pas besoin de porter des trucs)
		-- marche sur les PC (pas besoin de HW dédié -- ça revient un peu dans le 'pas cher')
		
		- pourquoi Unity (?)
		-- pour la gamification
		-- outil de prototypage rapide
		
		- pourquoi Zigfu (?)
		-- intégration facile avec Unity
		-- pas beaucoup de choix (basé sur le projet de l'OpenNI pas mis à jour depuis 2 ans maintenant...)
		--- <- include: délire sur pourquoi ce projet a été abandonnée (EVENTUELLMEENT)
		
		
		\subsection{Cahier de charges / user stories} 		---GEOFFREY--
		
		MAINTENANT QU'ON A FINI NOTRE ETUDE DE TERRAIN ON PROPOSE L'APPLICATION PARFAITE 
		
		ou pas - m'enfin bref.
		
		- faire les 4 exerices dont on a parlé
		-- pourquoi ces 4 là? Pourquoi pas les autres?
		--- problème pour les jambes (distances, bizarreté)
		--- problème pour mesurer la rotation (extrapolés)
		--- problème pour mesurer les doigts 
		
		- fournir un 'score' plus précis que celui que propose la FM
		-- l'angle exacte auquel on est arrivé
		-- savoir combien on a progrésé
		
		- tester si le patient fait un mouvement 'illegal'
		-- le prevenir
		-- recommencer le teste
		
		
		- gamification
		-- environnement agréeable
		-- ??? pourquoi ça s'applique mal
		--- mieux les truc répétifs (le FM on le fait genre une fois par semaine!)
		--- cela étant, les notions de feedback, d'intéractions 
			'meaningful play = integrated + discernable' (Rules of Play) les actions ont un 
			impact et on sait lequel. 
		
		\section{Mise en oeuvre} 		---WILLIAM + GEOFFREY--

		Grapher
		- FailGrapher (montre la croix en plus de la progression)
		
		Multicamera
		
		ExerciseMonitor
		- FlexionMontior et compagnie
		
		Machine à états (on aime bien les trucs visuels dans les rapports)
		
		\section{Limites}		---GEOFFREY--
		
		Fugl Meyer
		- c'est l'état l'art (golden stardard)
		
		Zigfu c'est de la merde!!!!
		- ils profitent d'un projet open source
		- ils apportent rien
		- pas de docs
		- code mal écrit
		- la moitié des samples marchent pas pour WILLIAM :D
		
		La Kinect en générale ??????
		- les rotations (pas récupérés par la SDK de Microsoft => déduits)
		  -- est il possible de les "voir" directement?
		- les mouvement distaux (pas de capture des doits individuelles)
		
		\section{Ouverture}		---GEOFFREY--
		
		- alternatifs: ARAT = test fonctionnelle et non déficience (le medecin il le vendait TRO')
		- projet de stage avec NaturalPad: faire de extraction de squelette moins nul
		- possibilités de concoursenzes avec Zigfu vu qu'il a juste repris 
		- la gamification est plus pertinenet / adapté aux exercices de rehabiliation 
		
\part{Annexes}

\end{document}