\documentclass[french,12pt]{report}
\usepackage[utf8]{inputenc}
\usepackage[francais]{babel}
\usepackage[T1]{fontenc}
\usepackage{lmodern}
\usepackage{ifpdf}
\usepackage{graphicx}
\usepackage{geometry}
\usepackage{color}
\usepackage{pdfpages}

\renewcommand{\familydefault}{\sfdefault}

\geometry{hmargin=50pt, vmargin=50pt}

\title{Rapport de projet : ZigFugl-Meyer}
\author{DYCE William \and MÉLIA Geoffrey}
\date{\today}
\ifpdf
	\pdfinfo 
	{
		/Author (wdyce, gmelia)
		/Title (Title (Rapport de projet))
		/Subject (Subject)
		/Keywords ()
		/CreationDate (\today)
	}
\fi

\begin{document}
	% Page de titre
	%\maketitle
	\thispagestyle{empty}
\begin{picture}(40,40)
\put(50,-600){\rule{.2mm}{21cm}}
\put(-50,-70){\rule{20cm}{.2mm}}

\put(70,-50){\textsc{\Huge{TER : ZigFugl-Meyer}}}
\put(360,-100){\textsc{\Large{Rapport de projet}}}

\put(100,-220){\textbf{\textsc{\large{Étudiants :}}}}
\put(100,-240){\textsc{\large{DYCE William, MÉLIA Geoffrey}}}

\put(100,-400){\textbf{\textsc{\large{Encadrant :}}}}
\put(170,-400){\textsc{\large{SEILLES Antoine}}}

\put(230,-530){\textsc{\large{Master informatique}}}
\put(205,-550){\textsc{\large{Année universitaire 2012-2013}}}
\end{picture}
	
	\thispagestyle{empty}
	\newpage
	
	% Sommaire
	\tableofcontents

	%Table des figures
	\listoffigures
	
	\newpage
	%TODO : grossir, centrer sur la page
	\section*{Remerciements}
	\addcontentsline{toc}{section}{Remerciements}
	\paragraph{}
	
\begin{itemize}
\item Antoine Seilles pour...
\item Ines Di Loreto
\item Julien Métrot et Karima Bakhti (et la jeune patiente) pour ...
\item Isabelle Laffont
\item Benoît Langes
\end{itemize}

\part{Rapport de projet}
\newpage
	\chapter{Introduction}
Les accidents vasculaires cérébraux (AVC) peuvent être la cause d'hémiplégie chez les personnes qui en sont victimes. La paralysie peut affecter une ou plusieurs parties du corps, jusqu'à être totale si la face, le tronc et les membres supérieurs et inférieurs sont paralysés. \\
La récupération des fonctions motrices, de la parole ou de la compréhension dépendent pour beaucoup de l'âge du patient et de son atteinte au niveau du cerveau.
		\section{Problématique}
Il existe de nombreux tests et échelles pour évaluer les capacités sensorimotrices de patients hémiplégiques [\ref {ref_analyse_litterature}]. Cependant, les médecins se retrouvent souvent confrontés au problème de la précision des mesures lors des test. Si la question ne se pose pas pour les tests dits "fonctionnels" (ex : le patient arrive t-il à se servir un verre d'eau?), des mesures d'angles et de positions se révèlent souvent nécessaires pour valider ou non la réussite d'un test par le patient.
\\Or, il existe peu d'outils pour réaliser ces mesures, et ceux-ci ne font pas l'unanimité. Pour la mesure d'angle entre les membres du patient (pli de l'épaule, du coude, etc.), le goniomètre se révèle être l'outil le plus utilisé, mais probablement par défaut (cf \ref{lapeyronie}). On lui reprochera en effet d'être: 
\begin{itemize}
	\item {intrusif :} Il doit être en contact direct avec le patient, pouvant fausser la mesure ou aider/gêner le patient.
	\item {imprécis :} L'épaisseur de peau et de graisse ne permet pas d'évaluer correctement l'angle formé par les os.
	\item {inutilisé :} Des médecins pourtant équipés vont préférer juger à l'œil, pour un meilleur ratio temps/précision.
\end{itemize}
		
		\subsubsection{Le test de Fugl-Meyer}
Cette imprécision des mesures devient dès lors gênante lorsque c'est un test non pas fonctionnel, mais de déficience qui est considéré comme le "Gold Standard" dans le domaine : le test de Fugl Meyer [\ref{fugl_meyer}]. En effet, celui-ci consiste à évaluer les mouvements du patient lors de la réalisation de gestes très précis, incluant des mouvements "éliminatoires", qu'il faut donc être capable de mesurer.	
	
	L'idée s'est alors posée d'utiliser un autre moyen de mesure, moins intrusif, pour la réalisation de tests de Fugl Meyer, largement utilisé dans le milieu de la réhabilitation, et de réfléchir aux possibilités d'enrichissement de celui-ci.
\newpage
		\section{Présentation du projet}
		
			\subsection{Étude de terrain}
		Le projet est avant tout une étude de faisabilité. Nous essayerons de voir
		à quel point nous pouvions procéder, avec un capteur de profondeur (en
		l'occurrence la Microsoft Kinect) et des bibliothèques et outils couramment 
		disponibles, à un test complet ou partiel de Fugl-Meyer.
		
		Nous étudierons donc d'une part les technologies disponibles pour la Kinect,
		et d'autre ce qu'est le test de Fugl-Meyer et quelles sont les 
		problématiques associées.
		
		\subsection{Prototype}
		Nous élaborerons ensuite une preuve de concept, c'est à dire un 
		prototype proposant au patient en cours de réhabilitation un test partiel 
		de Fugl-Meyer~:
		\begin{itemize}
		\item automatique,
		\item objectif, 
		\item ludique,
		\item granulaire.
		\end{itemize}
		
		\subsubsection{Automatisation et objectivité}
		Le système doit pouvoir procéder à une évaluation, de manière non-instrusive 
		et sans intervention externe, de ce qui se trouve devant lui. Il doit donc 
		proposer une série de notes objectives qui correspondent aux performances du 
		patient.
	
		\subsubsection{Aspect ludique~: "gamification"}
		La "gamification" est le processus qui consiste à ajouter de éléments 
		souvent associés au domaine du jeu (d'où le nom), comme des barres de
		progression ou des défis à relever, à une activité non-ludique. L'idée est 
		de favoriser l'implication et la motivation des participants. 
		
		Notons qu'il ne s'agit pas nécessairement de faire de l'activité en soi un 
		jeu.
		
		% Christopher Franklin "Errant Signal" on Gamification
		
    \subsubsection{Retour d'information continue et granulaire}
    La "granularité" d'un retour est le nombre d'options possibles. 
    Chaque exercice du Fugl-Meyer, par exemple, est noté sur trois points. Nous 
    aimerions proposer un alternatif plus nuancé.
    
    L'argument pour le retour d'information rejoint celui de la "gamification"~:
    la notion de "sens" dans un jeu, au moins d'après Salen et Zimmerman, est 
    directement lié à la richesse de ce retour.
    
    % "Rules of Play" on "meaningful play"

	\chapter{Background}
		\section{Kinect}

			\subsection{En bref}
La Kinect, connue à l'origine sous le nom "Project Natal" est une
periphérique informatique sortie par Microsoft, comportant une caméra
couleur, un microphone et des senseurs de profondeur à base de rayons
infra-rouges.

			\subsection{Une technologie couramment utilisé}
Sorti en Novembre 2010, le système était prévu à l'origine pour la console
de jeu de Microsoft, la "Xbox360". Comme la plupart des consoles et
périphériques de jeu elle
est donc vendu à perte, l'idée étant de récuppérér l'argent perdu en vendant
des jeux. Ceci fait de l'appareil le moins cher et le plus couramment
disponible de son genre.

			\subsection{Les bibliothèques officiels et tiers}
Il est peu surprenant que, dès sa sortie, des pilottes Libres et gratuites
ont étés crées à fin de permettre le detournement du périphérique par des
chercheurs et des amateurs. Microsoft a répondu en début de 2011 avec un
premier SDK pour Windows initialement peu complet. Des mises à jours et une
deuxième version, sortie en Octobre 2012, comblent ces lacqunes.

			\subsection{Bruits et signaux}
Il est important de noter que la recognition vocale et l'extraction de la
squelette ne sont pas faits par l'appareil lui-même~: ce sont des
bibliothèques sur le "client" qui s'en chargent. Il faut donc bien
différencier les pilottes, qui permettent d'accéder aux données brutes, des
bibliothèques de haut-niveau, qui interprétent ces données.


		
		\section{Fugl Meyer} \label{lapeyronie}		---GEOFFREY---
		
		PARLER DES DUDES QU'ON A RENCONTRE A PERLARONIE
		du goniomètre qui est tout pourri
		
		
		c'est un test qui teste des choses.
		
	\chapter{État de l'art} 		---WILLIAM---
	

	
		\section{Biblio Kinect}
		
		Drivers:
		MS drivers
		libfreenect
		
		Bibliothèques:
		OpenNI
		SDK Microsoft
		NITE de Primesense (racheté par MS ? )
		OpenCV
		FAAST
		Zigfu
		... etc
		
		
		\section{Interface naturelles appliqués aux mesures de mouvement 
		(ce titre, c'est de la merde - je vais le changer, mais plus tard...}

		NI + REHAB
		Voracy Fish (MOJOS)
		les truc qu'avec fait le LIRMM (MOJOS)
 		Natural Pad bien sur! Hammer \& Planks.
		-- exemples de Wii rehabilitation ?
		
		NI + MESURE DE PERFORMANCE
		Nike Kinect Training
		Wii Sport / Fit
		etcetera etcetera
		
		
	\chapter{Propositions}
	
	
	
		--> WHERE PUT DIS?????? Faire un ***SERIOUS*** jeu
		
		
		
		\section{Solutions}
		
		\subsection{Outils et technologies} 		---WILLIAM--
		- pourquoi la Kinect (?)
		-- c'est pas cher (trouver des exemples de trucs)
		-- c'est non intrusive (pas besoin de porter des trucs)
		-- marche sur les PC (pas besoin de HW dédié -- ça revient un peu dans le 'pas cher')
		
		- pourquoi Unity (?)
		-- pour la gamification
		-- outil de prototypage rapide
		
		- pourquoi Zigfu (?)
		-- intégration facile avec Unity
		-- pas beaucoup de choix (basé sur le projet de l'OpenNI pas mis à jour depuis 2 ans maintenant...)
		--- <- include: délire sur pourquoi ce projet a été abandonnée (EVENTUELLMEENT)
		
		
		\subsection{Cahier de charges / user stories} 		---GEOFFREY--
		
		MAINTENANT QU'ON A FINI NOTRE ETUDE DE TERRAIN ON PROPOSE L'APPLICATION PARFAITE 
		
		ou pas - m'enfin bref.
		
		- faire les 4 exerices dont on a parlé
		-- pourquoi ces 4 là? Pourquoi pas les autres?
		--- problème pour les jambes (distances, bizarreté)
		--- problème pour mesurer la rotation (extrapolés)
		--- problème pour mesurer les doigts 
		
		- fournir un 'score' plus précis que celui que propose la FM
		-- l'angle exacte auquel on est arrivé
		-- savoir combien on a progrésé
		
		- tester si le patient fait un mouvement 'illegal'
		-- le prevenir
		-- recommencer le teste
		
		
		- gamification
		-- environnement agréeable
		-- ??? pourquoi ça s'applique mal
		--- mieux les truc répétifs (le FM on le fait genre une fois par semaine!)
		--- cela étant, les notions de feedback, d'intéractions 
			'meaningful play = integrated + discernable' (Rules of Play) les actions ont un 
			impact et on sait lequel. 
		
		\section{Mise en oeuvre} 		---WILLIAM + GEOFFREY--

		Grapher
		- FailGrapher (montre la croix en plus de la progression)
		
		Multicamera
		
		ExerciseMonitor
		- FlexionMontior et compagnie
		
		Machine à états (on aime bien les trucs visuels dans les rapports)
		
		\section{Limites}		---GEOFFREY--
		
		Fugl Meyer
		- c'est l'état l'art (gold standard)
		- "effet plateau" - manque de granularité, possibilité d'avoir tous les points sans être "guéri".
		- variabilité des mesures selon le médecin (même les deux qu'on a vu...)	
		
		Zigfu c'est de la merde!!!!
		- ils profitent d'un projet open source
		- ils apportent rien
		- pas de docs
		- code mal écrit
		- la moitié des samples marchent pas pour WILLIAM :D
		
		La Kinect en générale ??????
		- les rotations (pas récupérés par la SDK de Microsoft => déduits)
		  -- est il possible de les "voir" directement?
		- les mouvement distaux (pas de capture des doits individuelles)
		
		\section{Ouverture}		---GEOFFREY--
		
		- alternatifs: ARAT = test fonctionnelle et non déficience (le medecin il le vendait TRO')
		- projet de stage avec NaturalPad: faire de extraction de squelette moins nul
		- possibilités de concoursenzes avec Zigfu vu qu'il a juste repris 
		- la gamification est plus pertinenet / adapté aux exercices de rehabiliation 
		
		\section{Références}
			\paragraph{Ressources web}
				\begin{itemize}
					\item 
				\end{itemize}
			\paragraph{Articles}
				\begin {itemize}
					\item : Evaluation of the disabilities of hemiplegic patients [M.-C. Gellez-Leman et al.] \label{ref_analyse_litterature}
				\end{itemize}
\part{Annexes}

\end{document}