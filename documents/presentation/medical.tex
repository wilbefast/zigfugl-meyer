\section{Aspect médical}

\begin{frame}
\tableofcontents[currentsection, hideothersubsections]
\end{frame}

\begin{frame}{Pré-requis}
	\begin{block}{Projet multidisciplinaire}
		\begin{itemize}
			\item Acquérir les connaissances médicales de base.
			\item Comprendre les enjeux et le contexte médico-social
			\item Analyser quels peuvent être les besoins
		\end{itemize}			
	\end{block}
		\begin{block}{Étude de terrain}
		\begin{itemize}
			\item Recherche et études documentaires
			\item Premières suppositions 
			\item Rencontrer patient et professionnels 
			\item Confronter les idées
		\end{itemize}			
	\end{block}
\end{frame}

\begin{frame}{Fugl-Meyer assessment Sensorimotor Recovery after Stroke}
	\begin{block}{Définition}
Mesure de la déficience moteur et sensorielle des membres supérieurs et inférieurs.
	\end{block}
	\begin{block}{Description}
Constitue un ensemble d'exercices dont chaque mouvement est évalué selon 1 ou plusieurs critères, permettant une évaluation de la sensibilité, du tonus, de la force et de la motricité du patient à travers l'analyse de son score.
	\end{block}
\end{frame}

\begin{frame}{}
	\begin{block}{Fonctionnement}
		Découpage en sous-sections pour chaque type de membres : distal, proximal, etc.
	\end{block}
\end{frame}