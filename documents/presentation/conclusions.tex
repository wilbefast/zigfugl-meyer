\section{Conclusions}

\begin{frame}{Conclusions}
\only<1-2>
{
  \begin{block}{Résultats}
  \begin{itemize}
  \item Test automatique,
  \item Retour granulaire,
  \item Aspect ludique qui émerge du retour d'informations.
  \end{itemize}
  \end{block}
}
\only<2>
{
  \begin{alertblock}{Problèmes}
  \begin{itemize}
  \item Trop de temps passé dans les menus,
  \item Le patient regarde le membre,
  \item Le retour incite à forcer.
  \end{itemize}
  \end{alertblock}
}
\only<3-4>
{
  \begin{alertblock}{Limites}
  \begin{itemize}
  \item Précision insuffisante,
  \item Rotations déduites des positions,
  \item Zigfu.
  \end{itemize}
  \end{alertblock}
}
\only<4>
{
  \begin{block}{Ouverture}
  \begin{itemize}
  \item Poursuite de l'étude en stage~:
  \begin{itemize}
  \item \textbf{Geoffrey~:} paramétrage thérapeutique de jeu,
  \item \textbf{William~:} meilleur extraction de squelette. 
  \end{itemize}
  \end{itemize}
  \end{block}
}
\end{frame}